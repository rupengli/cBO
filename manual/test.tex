\documentclass[11pt]{article}
\usepackage[margin=1in]{geometry}
\usepackage{amsmath, amssymb}
\usepackage{booktabs}
\usepackage{hyperref}

\title{Constrained Bayesian Optimization Testbed: Egg Model (OPM Flow)}
\date{}
\begin{document}
\maketitle

\section{Overview}
We build a constrained Bayesian optimization (CBO) test environment using the \emph{Egg} waterflood benchmark and the \emph{OPM Flow} black-oil reservoir simulator. Each function evaluation runs the simulator once and returns:
\begin{itemize}
  \item a scalar objective (NPV computed from cumulative totals), and
  \item several black-box constraints computed from simulator outputs (facility/capacity constraints).
\end{itemize}

\section{Getting the code (GitHub)}
Clone the project repository:
\begin{verbatim}
git clone https://github.com/rupengli/cBO.git
\end{verbatim}
Then change into the repository folder:
\begin{verbatim}
cd cBO
\end{verbatim}

\section{Downloading the Egg model data files}
The Egg benchmark deck and supporting include files are available from:
\begin{center}
\url{https://data.4tu.nl/articles/dataset/The_Egg_Model_-_data_files/12707642}
\end{center}
Download the dataset zip (e.g., \texttt{data.zip}) and unzip it in the project folder:
\begin{verbatim}
unzip data.zip
\end{verbatim}
After unzipping, ensure the folder layout matches what the code expects. By default, the config uses:
\[
\texttt{paths.dataset\_root: Egg\_Model\_Data\_Files\_v2}
\]
and the evaluator expects (at minimum) paths such as:
\begin{itemize}
  \item \texttt{Egg\_Model\_Data\_Files\_v2/Eclipse/Egg\_Model\_ECL.DATA}
  \item \texttt{Egg\_Model\_Data\_Files\_v2/Eclipse/SCHEDULE\_NEW.INC}
  \item \texttt{Egg\_Model\_Data\_Files\_v2/Permeability\_Realizations/PERM1\_ECL.INC} (and other realizations)
\end{itemize}
If you unzip into a different folder name or location, update \texttt{paths.dataset\_root} in the YAML/JSON config accordingly.

\section{Installing OPM Flow}
Install OPM Flow by following the official OPM instructions:
\begin{center}
\url{https://opm-project.org/?page_id=36}
\end{center}
Verify that the \texttt{flow} executable is available in your shell (\texttt{flow --version}).

\section{Quick start (run one evaluation)}
With OPM Flow installed and the data files unzipped, run:
\begin{verbatim}
python3 cbo_flow/egg_main.py --config cbo_flow/egg_config.yaml
\end{verbatim}
This will:
\begin{itemize}
  \item create a fresh run directory (\texttt{paths.work\_dir}),
  \item run \texttt{flow} on the patched deck, and
  \item print MATLAB-style outputs (objective + constraints) and write plots if enabled.
\end{itemize}

\section{Decision Variables (8$\times$stages, piecewise-constant controls)}
The Egg setting uses 12 wells: 8 injectors and 4 producers. The decision variables control only the \textbf{injector rates}. Producer BHP is fixed at a constant value (395 bar in the current config). Controls are \emph{piecewise-constant} over $n_{\mathrm{stages}}$ stages on the full horizon $[0,T]$. The number of stages is a user input in the config file.

\subsection*{Multi-stage usage (any $8n_{\mathrm{stages}}$ problem)}
You can choose \emph{any} number of stages $n_{\mathrm{stages}}\ge 1$. The only requirement is:
\[
\text{len}(\texttt{controls.vars}) = 8n_{\mathrm{stages}}.
\]
The list is ordered stage-by-stage:
\begin{itemize}
  \item Stage 1 uses the first 8 values: \texttt{INJECT1..INJECT8}
  \item Stage 2 uses the next 8 values: \texttt{INJECT1..INJECT8}
  \item \dots
  \item Stage $n_{\mathrm{stages}}$ uses the last 8 values
\end{itemize}

\subsection*{How stages map to time (\texttt{stage\_tsteps})}
The simulator deck already contains a fixed number of time-step blocks (in this Egg setup it is 120 blocks). The config entry \texttt{controls.stage\_tsteps} specifies how many of those blocks are allocated to each stage:
\begin{itemize}
  \item It must have length $n_{\mathrm{stages}}$
  \item Its elements must sum to the total number of blocks (120 for the provided Egg schedule)
\end{itemize}
If \texttt{stage\_tsteps} is omitted, the code splits the blocks evenly across stages.

\paragraph{Example: 2 stages.}
\begin{verbatim}
controls:
  n_stages: 2
  vars: [qI1..qI8_stage1, qI1..qI8_stage2]   # 16 values total
  stage_tsteps: [60, 60]                     # first half, second half
\end{verbatim}

\paragraph{Example: 3 stages.}
\begin{verbatim}
controls:
  n_stages: 3
  vars: [8 values for stage1, 8 for stage2, 8 for stage3]  # 24 total
  stage_tsteps: [40, 40, 40]                                # equal split
\end{verbatim}

\paragraph{Example: 4 stages (non-uniform).}
\begin{verbatim}
controls:
  n_stages: 4
  vars: [32 values total]                 # 8*4
  stage_tsteps: [10, 20, 30, 60]          # sums to 120
\end{verbatim}

\subsection*{Control policy}
\begin{itemize}
  \item \textbf{Injectors (8 wells):} control water injection \textbf{rates} in each stage.
  \item \textbf{Producers (4 wells):} producer \textbf{BHP} is fixed (not optimized).
\end{itemize}

Let injectors be $I_1,\dots,I_8$ and producers be $P_1,\dots,P_4$. The decision vector
\[
x \in \mathbb{R}^{8n_{\mathrm{stages}}}
\]
is defined as a flat concatenation, \emph{stage by stage}:
\[
x^{(s)} =
\big(
q_{I_1}^{(s)},\dots,q_{I_8}^{(s)}
\big),
\qquad
x = \big(x^{(1)},\dots,x^{(n_{\mathrm{stages}})}\big),
\]
where $q_{I_j}^{(s)}$ is the injection rate for injector $I_j$ in stage $s$.

\begin{table}[h]
\centering
\begin{tabular}{llll}
\toprule
Variable block & Wells & Control type & Count \\
\midrule
$\{q_{I_j}^{(s)}\}_{j=1}^8$ & Injectors $I_1\ldots I_8$ & Rate (per stage) & $8$ \\
\midrule
Total & 12 wells & $n_{\mathrm{stages}}$ stages & $8n_{\mathrm{stages}}$ \\
\bottomrule
\end{tabular}
\caption{Decision variables for the Egg CBO testbed.}
\end{table}

\section{Objective (Black-box NPV from cumulative totals)}
We use a simple, constant-economics NPV with \emph{no discounting}, computed from end-of-horizon cumulative totals returned by the simulator:
\begin{equation}
\label{eq:npv}
\mathrm{NPV}(x)
=
p_o \cdot \mathrm{CumOil}(T)
-
c_w \cdot \mathrm{CumWaterProd}(T)
-
c_{inj} \cdot \mathrm{CumWaterInj}(T),
\end{equation}
where:
\begin{itemize}
  \item $\mathrm{CumOil}(T)$: field cumulative oil produced,
  \item $\mathrm{CumWaterProd}(T)$: field cumulative water produced,
  \item $\mathrm{CumWaterInj}(T)$: field cumulative water injected,
  \item $p_o>0$: oil price (constant),
  \item $c_w>0$: produced-water handling cost (constant),
  \item $c_{inj}>0$: injected-water cost (constant).
\end{itemize}
If the optimizer is in minimization form, we use:
\[
f(x) = -\mathrm{NPV}(x).
\]
The mapping $x \mapsto \mathrm{NPV}(x)$ is black-box since it requires a full simulator run.

\section{Constraints (Black-box, output-based)}
We enforce facility/capacity constraints computed from simulator output time series.
All constraints are of the form $g_i(x)\le 0$.

\subsection{C1: Peak field water injection rate (capacity)}
Let $q_{inj}^{field}(t)$ be the field total water injection rate. Define:
\begin{equation}
g_{\mathrm{winj}}(x)
=
\max_{t\in[0,T]}
q_{inj}^{field}(t)
-
Q_{\mathrm{winj}}^{\max}
\le 0.
\end{equation}
\paragraph{Parameter guidance.}
$Q_{\mathrm{winj}}^{\max}$ is a surface facility/injection capacity. In many engineering scenarios it is set high enough that this constraint is rarely violated (often effectively ``loose''). If you want C1 to be active in optimization, choose $Q_{\mathrm{winj}}^{\max}$ closer to typical peak injection (e.g., calibrate from a baseline run) rather than an extremely large value.

\subsection{C2: Peak field water production rate (treatment/disposal)}
\begin{equation}
g_{\mathrm{wprod}}(x)
=
\max_{t\in[0,T]}
q_w^{field}(t)
-
Q_{\mathrm{wprod}}^{\max}
\le 0.
\end{equation}
\paragraph{Parameter guidance.}
$Q_{\mathrm{wprod}}^{\max}$ represents produced-water treatment/disposal capacity. It is often set conservatively high and therefore rarely violated. To make C2 meaningful, set it closer to expected peak produced-water rates; otherwise most solutions will satisfy it (``weak'' constraint).

\subsection{C3: Peak field liquid production rate (platform liquid handling)}
Let $q_o^{field}(t)$ and $q_w^{field}(t)$ be the field oil and water production rates. Define:
\begin{equation}
g_{\mathrm{liq}}(x)
=
\max_{t\in[0,T]}
\big(q_o^{field}(t)+q_w^{field}(t)\big)
-
Q_{\mathrm{liq}}^{\max}
\le 0.
\end{equation}
\paragraph{Parameter guidance.}
$Q_{\mathrm{liq}}^{\max}$ is the platform liquid-handling capacity. Like C1--C2, it can be very loose if set far above the expected peak liquid rate. If you want this constraint to filter a non-trivial fraction of candidate controls, set it near the baseline peak liquid rate (or only slightly above it).

\subsection{C4: Peak field water-cut (guardrail)}
Define field water cut:
\[
\mathrm{WC}(t) = \frac{q_w^{field}(t)}{q_o^{field}(t)+q_w^{field}(t)+\epsilon},
\]
with a small $\epsilon>0$ to avoid division by zero. The constraint is:
\begin{equation}
g_{\mathrm{wc}}(x)
=
\max_{t\in[0,T]}
\mathrm{WC}(t)
-
\mathrm{WC}^{\max}
\le 0.
\end{equation}
\paragraph{Parameter guidance.}
$\mathrm{WC}^{\max}$ is typically the most ``active'' operational guardrail: the \emph{lower} $\mathrm{WC}^{\max}$ is, the \emph{more restrictive} the constraint.
For practical engineering scenarios, $\mathrm{WC}^{\max}\approx 0.8$ is a reasonable lower bound and can remove a substantial portion of solutions; values around $0.6$ are very restrictive; values below $0.6$ are unusual. Values higher than about $0.95$ are also uncommon and usually make C4 almost irrelevant.

\subsection{C5: Early breakthrough (field-level water cut up to $T_{\mathrm{early}}$ years)}
Define the same field water cut $\mathrm{WC}(t)$ as above. For a user-chosen early window $T_{\mathrm{early}}$ (years) and a small threshold $\mathrm{WC}_\epsilon$, the constraint is:
\begin{equation}
g_{\mathrm{wc,early}}(x)
=
\max_{t \le T_{\mathrm{early}}}\mathrm{WC}(t) - \mathrm{WC}_\epsilon \le 0.
\end{equation}
\paragraph{Parameter guidance.}
C5 enforces ``oil-only'' production early in life: it requires field water cut to remain below $\mathrm{WC}_\epsilon$ up to $T_{\mathrm{early}}$.
Typical settings for $T_{\mathrm{early}}$ are 1--3 years; values above about 4 years are very restrictive. Smaller $\mathrm{WC}_\epsilon$ (e.g., 0.01) further tightens the constraint.

\subsection{C6: Breakthrough-time spread across producers (well-level)}
Define per-producer water cut:
\[
\mathrm{WC}_j(t) = \frac{q_{w,j}(t)}{q_{o,j}(t)+q_{w,j}(t)+\epsilon},
\]
using producer oil/water rates. Define breakthrough time for producer $j$:
\[
t_{bt,j}=\min\{t:\mathrm{WC}_j(t)\ge \mathrm{WC}_\epsilon\},
\]
and set $t_{bt,j}=+\infty$ if it never crosses within the horizon. The constraint is:
\begin{equation}
g_{\mathrm{bt,spread}}(x)
=
\big(\max_j t_{bt,j}-\min_j t_{bt,j}\big) - \Delta t_{\max} \le 0.
\end{equation}
\paragraph{Parameter guidance.}
$\Delta t_{\max}$ controls how synchronized producer breakthrough must be.
Large values (e.g., 6 years) are loose and most solutions will satisfy them; values around 1--2 years are very restrictive. This constraint is useful for avoiding strategies where one producer breaks through very early while others remain ``clean''.

\subsection*{Constraint names and residuals (exactly as printed by the code)}
The evaluator prints MATLAB-style residuals with these exact names (see \texttt{cbo\_flow/egg\_main.py}). Each printed value is a constraint residual of the form:
\[
C_k(x) = \text{metric}(x) - \text{limit},
\]
and the constraint is satisfied if $C_k(x)\le 0$ (negative means slack).
\begin{itemize}
  \item \texttt{C1\_PeakFieldWaterInjectionRate}: peak field injected water rate minus limit (SM3/DAY).\\
  Definition: $\max_t(\mathrm{FWIR}(t)) - Q_{\mathrm{winj}}^{\max} \le 0$.
  \item \texttt{C2\_PeakFieldWaterProductionRate}: peak field produced water rate minus limit (SM3/DAY).\\
  Definition: $\max_t(\mathrm{FWPR}(t)) - Q_{\mathrm{wprod}}^{\max} \le 0$.
  \item \texttt{C3\_PeakFieldLiquidRate}: peak field liquid rate minus limit (SM3/DAY).\\
  Definition: $\max_t(\mathrm{FOPR}(t)+\mathrm{FWPR}(t)) - Q_{\mathrm{liq}}^{\max} \le 0$.
  \item \texttt{C4\_PeakFieldWaterCut}: peak field water cut minus limit (-).\\
  Definition: $\max_t\!\left(\frac{\mathrm{FWPR}(t)}{\mathrm{FOPR}(t)+\mathrm{FWPR}(t)+\epsilon}\right) - \mathrm{WC}^{\max} \le 0$.
  \item \texttt{C5\_EarlyWaterCutUpToYears}: early breakthrough constraint on field water cut (-).\\
  Definition: $\max_{t\le T_{\mathrm{early}}}(\mathrm{WC}(t)) - \mathrm{WC}_\epsilon \le 0$.
  \item \texttt{C6\_BreakthroughTimeSpreadAcrossProducers}: producer breakthrough-time spread constraint (years).\\
  Definition: $(\max_j t_{bt,j}-\min_j t_{bt,j}) - \Delta t_{\max} \le 0$ where $t_{bt,j}$ is the first time $\mathrm{WC}_j(t)\ge \mathrm{WC}_\epsilon$.
\end{itemize}

\section{How thresholds are configured (matches the code)}
All ``hyperparameters'' live in the YAML/JSON config (no hard-coded values in the evaluator). In particular:
\begin{itemize}
  \item \texttt{constraints.C1\_PeakFieldWaterInjectionRate.limit} sets $Q_{\mathrm{winj}}^{\max}$.
  \item \texttt{constraints.C2\_PeakFieldWaterProductionRate.limit} (or \texttt{limit\_factor}) sets $Q_{\mathrm{wprod}}^{\max}$.
  \item \texttt{constraints.C3\_PeakFieldLiquidRate.limit} (or \texttt{limit\_factor}) sets $Q_{\mathrm{liq}}^{\max}$.
  \item \texttt{constraints.C4\_PeakFieldWaterCut.limit} sets $\mathrm{WC}^{\max}$.
  \item \texttt{constraints.C5\_EarlyWaterCutUpToYears.years} sets $T_{\mathrm{early}}$.
  \item \texttt{constraints.C5\_EarlyWaterCutUpToYears.wc\_eps} sets $\mathrm{WC}_\epsilon$.
  \item \texttt{constraints.C6\_BreakthroughTimeSpreadAcrossProducers.delta\_years\_max} sets $\Delta t_{\max}$.
\end{itemize}

\section{Black-box evaluation interface}
A single evaluation takes $x$ and returns objective and constraints:
\[
\texttt{evaluate}(x)\ \mapsto\ \big(f(x),\ g_{\mathrm{winj}}(x),\ g_{\mathrm{wprod}}(x),\ g_{\mathrm{liq}}(x),\ g_{\mathrm{wc}}(x),\ g_{\mathrm{wc,early}}(x),\ g_{\mathrm{bt,spread}}(x)\big),
\]
where the simulator run and output parsing are encapsulated as a black box.

\section{Implementation notes (what the code actually does)}
The current implementation is driven by \texttt{cbo\_flow/egg\_main.py} and a YAML/JSON config (default: \texttt{cbo\_flow/egg\_config.yaml}):
\begin{itemize}
  \item It creates a fresh work directory (copies the base deck, patches controls and optionally a permeability realization).
  \item It runs \texttt{flow Eclipse/Egg\_Model\_ECL.DATA --output-dir=out\_serial}.
  \item It parses the Eclipse-style tables in the \texttt{*.PRT} to compute NPV and peak field-rate constraints (C1--C4).
  \item It reads \texttt{*.SMSPEC/*.UNSMRY} (via \texttt{resdata}) to compute well-level breakthrough times and C5--C6, and to plot producer water cuts.
\end{itemize}

\subsection*{Plots}
The plotting is done by \texttt{cbo\_flow/make\_plots.py}. When enabled in the config:
\begin{itemize}
  \item \texttt{rates\_prt.png}: field rates (and optional water cut) from PRT.
  \item \texttt{WC\_PROD.png}: one plot with four curves (PROD1--PROD4) for producer water cut.
\end{itemize}

\end{document}
